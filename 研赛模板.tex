% !Mode:: "TeX:UTF-8"
%!TEX program  = xelatex

\documentclass[bwprint]{gmcmthesis}

\title{题目}
\baominghao{191024000000} %参赛队号
\schoolname{上海交通大学}%学校名称
\membera{同学1} %队员A
\memberb{同学2} %队员B
\memberc{同学3} %队员C
\begin{document}
 
 %生成标题
 \maketitle
 
 %填写摘要
\begin{abstract}

论文摘要。
注意篇幅一般不超过两页,且无需译成英文。
全国评阅时对摘要和论文都会审阅。
   

\keywords{关键词1\quad 关键词2\quad  关键词3\quad  关键词4\quad 关键词5}
\end{abstract}

\pagestyle{plain}

%目录 不推荐加
\tableofcontents
\newpage
\section{问题重述}

\subsection{引言}


问题引入继续。这些一级二级标题请自定义修改。

\subsection{问题的提出}
问题的提出这些一级二级标题请自定义修改。
示例图:
\begin{figure}[!h]
\centering
\includegraphics[width=.7\textwidth]{testfig.png}
\caption{图}
\end{figure}

问题提出继续。这些一级二级标题请自定义修改。

\section{已知信息}
\subsection{标题1}
解释。这些一级二级标题请自定义修改。

\subsection{标题2}
解释。这些一级二级标题请自定义修改。

\subsection{标题3}

解释这些一级二级标题请自定义修改。

如需图:

 \begin{figure}[!h]
\centering
\includegraphics[width=.9\textwidth]{testfig.png}
 \end{figure}




 \begin{figure}[!h]
\centering
\includegraphics[width=.9\textwidth]{testfig.png}
 \end{figure}

\section{模型的假设}

这些一级二级标题请自定义修改。
\begin{itemize}
\item 假设1
\item 假设2
\item 假设3
\item 假设4
\item 假设5

\end{itemize}

\section{符号说明(参考制表)}

\begin{tabular}{cc}
 \hline
 \makebox[0.4\textwidth][c]{符号}	&  \makebox[0.5\textwidth][c]{意义} \\ \hline
 n(t)	    & t时刻数目1 \\ \hline
 L(t)	    & t时刻数目2  \\ \hline
 W(i)	    & 第i个gate占用数目 \\ \hline
 s       &  卫星厅的登机口使用数目 \\ \hline
t    &      T航站楼使用的登机口数目 \\ \hline
$s_{ii}$&国际到达国际出发的S卫星厅登机口数目 \\ \hline
$s_{di}$&国内到达国际出发的S卫星厅登机口数目 \\ \hline
 $s_{id}$&国际到达国内出发的S卫星厅登机口数目 \\ \hline
$s_{dd}$&国内到达国内出发的S卫星厅登机口数目 \\ \hline
$t_{ii}$&国际到达国际出发的T航站楼登机口数目 \\ \hline
$t_{dd}$&国内到达国内出发的T航站楼登机口数目 \\ \hline
$t_{id}$&国际到达国内出发的T航站楼的登机口数目 \\ \hline
$t_{di}$&国内到达国际出发的T航站楼的登机口数目 \\ \hline
\end{tabular}

\newpage
\section{问题分析}
这些一级二级标题请自定义修改。
\subsection{问题一模型建立及求解}
这些一级二级标题请自定义修改。
\subsubsection{问题分析}
这些一级二级标题请自定义修改。
问题解析。



\begin{figure}[!h]
\centering
\includegraphics[width=.9\textwidth]{testfig.png}
\caption{图的名字}
\end{figure}


问题解析。

\subsubsection{模型建立与求解}
这些一级二级标题请自定义修改。
优化目标:

 (1) ;

 (2) 。

约束条件:

 (1) 

 (2) 

 (3) 

\begin{figure}[!h]
\centering
\includegraphics[width=.9\textwidth]{testfig.png}
\caption{问题一求解算法流程图}
\end{figure}

这些一级二级标题请自定义修改。
求解过程:

具体解释。

问题解决的解释。

\begin{figure}[!h]
\centering
\includegraphics[width=.9\textwidth]{testfig.png}
\caption{图的名字}
\end{figure}
\newpage

\subsection{问题二模型建立及求解}

\subsubsection{问题分析}

问题2的问题解析。

\subsubsection{模型建立与求解}

优化目标:

 (1)目标1

 (2)目标2

约束条件:
 (1)条件1

 (2)条件2

 (3)条件3

 (4)条件4
 

\begin{figure}[!h]
\centering
\includegraphics[width=.9\textwidth]{testfig.png}
\caption{问题二求解算法流程图}
\end{figure}

求解过程:
具体解释等。

\begin{figure}[!h]
\centering
\includegraphics[width=.9\textwidth]{testfig.png}
\caption{图的名字}
\end{figure}

解释方案。

\begin{figure}[!h]
\centering
\includegraphics[width=.9\textwidth]{testfig.png}
\caption{图的名字}
\end{figure}
\newpage
\subsection{问题三模型建立及求解}

问题三的模型建立以及具体解决方案。


\subsubsection{问题分析}

在第二问的最优解的情况下,解释清楚问题
 

\subsubsection{模型建立及求解}

建立的模型以及结果如图所示。


\begin{figure}[!h]
\centering
\includegraphics[width=.9\textwidth]{testfig.png}
\caption{图的名字}
\end{figure}


\newpage
%\section{参考文献}
%参考文献   手工录入
\begin{thebibliography}{9}%宽度9
 \bibitem{bib:one} 张利宁,邱涤珊,李皓平,黄小军. (2010). 基于模型分解的多机带时间窗口任务规划算法. 计算机应用, 30(11).
 \bibitem{bib:two}石荣,蒋凤伟,李明截. (2010). 图论最大流理论在机场登机口分配中的应用. 中国民航大学学报, 28(5).
 \bibitem{bib:three}廖正文,苗建瑞,孟令云,李海鹰,赵岚. (2016). 基于拉格朗日松弛的双线铁路列车运行图优化算法. 铁道学报, 38(9).
\end{thebibliography}

%采用bibtex方案
 


\newpage
%附录
\appendix
%\setcounter{page}{1} %如果需要可以自行重置页码。
\section{源程序}
\begin{lstlisting}[language=python]%设置不同语言即可。

代码1 内容

 \end{lstlisting}

\begin{lstlisting}[language=perl]%设置不同语言即可。

代码2 内容
 \end{lstlisting}
\end{document} 